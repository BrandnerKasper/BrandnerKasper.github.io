%!TEX TS-program = xelatex
%!TEX encoding = UTF-8 Unicode
% Awesome CV LaTeX Template for Cover Letter
%
% This template has been downloaded from:
% https://github.com/posquit0/Awesome-CV
%
% Authors:
% Claud D. Park <posquit0.bj@gmail.com>
% Lars Richter <mail@ayeks.de>
%
% Template license:
% CC BY-SA 4.0 (https://creativecommons.org/licenses/by-sa/4.0/)
%


%-------------------------------------------------------------------------------
% CONFIGURATIONS
%-------------------------------------------------------------------------------
% A4 paper size by default, use 'letterpaper' for US letter
\documentclass[11pt, a4paper]{awesome-cv}

% Configure page margins with geometry
\geometry{left=1.4cm, top=.8cm, right=1.4cm, bottom=1.8cm, footskip=.5cm}

% Specify the location of the included fonts
\fontdir[fonts/]

% Color for highlights
% Awesome Colors: awesome-emerald, awesome-skyblue, awesome-red, awesome-pink, awesome-orange
%                 awesome-nephritis, awesome-concrete, awesome-darknight
% \colorlet{awesome}{awesome-red}
% Uncomment if you would like to specify your own color
\definecolor{awesome}{HTML}{74b71a}

% Colors for text
% Uncomment if you would like to specify your own color
% \definecolor{darktext}{HTML}{414141}
% \definecolor{text}{HTML}{333333}
% \definecolor{graytext}{HTML}{5D5D5D}
% \definecolor{lighttext}{HTML}{999999}

% Set false if you don't want to highlight section with awesome color
\setbool{acvSectionColorHighlight}{true}

% If you would like to change the social information separator from a pipe (|) to something else
\renewcommand{\acvHeaderSocialSep}{\quad\textbar\quad}


%-------------------------------------------------------------------------------
%	PERSONAL INFORMATION
%	Comment any of the lines below if they are not required
%-------------------------------------------------------------------------------
% Available options: circle|rectangle,edge/noedge,left/right
% \photo[circle,noedge,left]{./examples/profile}
\name{Tobias}{Brandner}
\position{MSc Computer Science{\enskip\cdotp\enskip}Specialization in Artificial Intelligence}
\address{Semmelstraße 9, 97070 Würzburg, Bavaria - Germany}

\mobile{(+49) 179 829 8854}
\email{tobias.brandner@gmx.de}
\homepage{brandnerkasper.github.io}
\github{BrandnerKasper}
\linkedin{tobias-brandner}
% \gitlab{gitlab-id}
% \stackoverflow{SO-id}{SO-name}
% \twitter{@twit}
% \skype{skype-id}
% \reddit{reddit-id}
% \medium{madium-id}
% \googlescholar{googlescholar-id}{name-to-display}
%% \firstname and \lastname will be used
% \googlescholar{googlescholar-id}{}
% \extrainfo{extra informations}

% \quote{``There is a difference between forgetting something and never having learned it." - Dr. Frederic P. Schuller}


%-------------------------------------------------------------------------------
%	LETTER INFORMATION
%	All of the below lines must be filled out
%-------------------------------------------------------------------------------
% The company being applied to
\recipient
  {NVIDIA PerfTech}
  {Nvidia Corporation\\ Germany}
% The date on the letter, default is the date of compilation
\letterdate{\today}
% The title of the letter
\lettertitle{Job Application for Tools Engineer - Games and AI - ID JR1994688}
% How the letter is opened
\letteropening{Dear PerfTech Team,}
% How the letter is closed
\letterclosing{Sincerely,}
% Any enclosures with the letter
\letterenclosure[Attached]{Curriculum Vitae}


%-------------------------------------------------------------------------------
\begin{document}

% Print the header with above personal informations
% Give optional argument to change alignment(C: center, L: left, R: right)
\makecvheader[R]

% Print the footer with 3 arguments(<left>, <center>, <right>)
% Leave any of these blank if they are not needed
\makecvfooter
  {\today}
  {Tobias Brandner~~~·~~~Cover Letter}
  {}

% Print the title with above letter informations
\makelettertitle

%-------------------------------------------------------------------------------
%	LETTER CONTENT
%-------------------------------------------------------------------------------
\begin{cvletter}

\lettersection{About Me}
Have you ever wondered how a video game works under the hood? 
How a single frame is rendered?  
% How game objects interact?
 Why the computational workload varies with scene complexity and resolution
 —and how we can optimize it in unconventional ways? 
 These questions sparked my curiosity and led me to pursue a BSc in Games Engineering, 
 followed by an MSc in Computer Science with a focus on machine learning. 
 Unfortunately, this also means I can no longer unsee artifacts like aliasing while playing games!

\lettersection{Why NVIDIA?}
The desire to offer players the best possible gaming experience while leveraging cutting-edge technology appeals to me. 
NVIDIA stands at the forefront of this field, 
with AI-powered advancements like Deep Learning Super Sampling (DLSS) for example. 
Innovations like these inspired me to explore neural rendering in the context of Unreal Engine 5 as my Master's thesis.

Beyond gaming, AI-driven tools and performance optimization have become key interests of mine. 
NVIDIA’s approach—bridging hardware and software teams to create state-of-the-art developer tools—
aligns well with my expertise and ambitions. 
Joining the PerfTech team would be an exciting opportunity to contribute to profiling and optimization 
analysis tools, working closely with both hardware and software engineers to 
push gaming and AI applications to the next level.

\lettersection{Why Me?}

% - Job traits: math kernels for hardware accelerators, PyTorch, C, C++, and Python, 
%               Collaborating with your team, detailed documentation, Identifying and resolving technical challenges

% Whether it’s implementing computer vision models in PyTorch, 
% contributing to Unreal Engine projects that led to a publication at the IEEE Conference on Games, 
% or teaching game engine development in C++ with OpenGL, I enjoy tackling complex technical challenges 
% and sharing knowledge. My problem-solving approach focuses on breaking down tasks into manageable steps 
% and seeking input from hardware and software teams to ensure efficient, well-integrated solutions.

During my master’s thesis, I focused on real-time rendering super-resolution, 
specifically scaling from 1080p to 4K at 60 FPS. 
This required choosing networks with inference times under 16ms and VRAM usage below 1GB.
To streamline model evaluation, 
I designed each model file to run independently with dummy tensors. 
This approach allowed me to debug architectures faster, measure VRAM usage and inference speed in isolation, 
and assess suitability without integrating them into the full pipeline. 
Though simple, this idea significantly improved my workflow and decision-making.

This experience reinforced my belief that even the smallest ideas and tools can have a massive impact on productivity. 
I’d love the opportunity to bring this mindset to your team and contribute to your projects. 
Looking forward to hearing from you!
\end{cvletter}


%-------------------------------------------------------------------------------
% Print the signature and enclosures with above letter informations
\makeletterclosing

\end{document}
