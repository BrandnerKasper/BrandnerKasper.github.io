%-------------------------------------------------------------------------------
%	SECTION TITLE
%-------------------------------------------------------------------------------
\cvsection{Projects}


%-------------------------------------------------------------------------------
%	CONTENT
%-------------------------------------------------------------------------------
\begin{cventries}

%--------------------------------------------------------- \faGamepad
  \cventry
    {Computer Vision, Pytorch, Python, Unreal Engine 5} % Affiliation/role
    {Neural Rendering - Unreal Real-Time Rendering Super Resolution} % Organization/group
    {\href{https://github.com/BrandnerKasper/URTSR}{\faGithubSquare}} % HyperLink github
    {Jan. 2024 - Sep. 2024} % Date(s)
    {
      \begin{cvitems} % Description(s) of experience/contributions/knowledge
        \item {Developed a neural network to increase the resolution (1080p to 4k) as well as mitigating artefacts (e.g. aliasing) of rendered frames in real-time.}
        \item {Generated and labeled a new dataset (around 500GB) using Unreal Engine 5, containing different animated 3d-person characters traversing four different high fidelity environments.}
        \item {Designed and iterated on the neural network architecture (CNN, U-Net and ViT) in Pytorch (Python).}
        \item {Compared results to other implemented SOTA methods on image quality metrics (PSNR, SSIM and LPIPS), VRAM usage and inference speed.}
      \end{cvitems}
    }

%---------------------------------------------------------
  \cventry
    {Virtual Reality, Unreal Engine 5} % Affiliation/role
    {Abyssal Engima - Dive In Edition (VR)} % Organization/group
    {\href{https://miggli.itch.io/abyssal-enigma}{\faGamepad}} % HyperLink itchio
    {May 2023 - Nov. 2023} % Date(s)
    {
      \begin{cvitems} % Description(s) of experience/contributions/knowledge
        \item {Collaborated on a first person deep sea exploration game made in Unreal Engine 5 in a team of 6 people, including writer, designer and artists.}
        \item {Implemented the first person controller, including animations, particle/audio effects and in-game cinematic.}
        \item {Ported the game to Virtual Reality (VR), incorporating anti-motion sickness techniques, e.g. virtual nose.}
      \end{cvitems}
    }

%---------------------------------------------------------
  \cventry
    {Computer Vision, Pytorch, Python} % Affiliation/role
    {Multi Language Image Classification} % Organization/group
    {\href{https://github.com/BrandnerKasper/MP_CustomCoOp}{\faGithubSquare}} % HyperLink github
    {Apr. 2023 - Sep. 2023} % Date(s)
    {
      \begin{cvitems} % Description(s) of experience/contributions/knowledge
        \item {Modified existing vision-language classification models to explore them in a multi-lingual context.}
        \item {Extended a Python/Pytorch code base with other open-source pre-trained vision-language models (Roberta-ViT-B32) and few-shot trained them on Caltech101 dataset.}
        \item {Analyzed the results to increase the models accuracy on more difficult multi-lingual tasks.}
      \end{cvitems}
    }

%---------------------------------------------------------
  \cventry
    {Unreal Engine 5, C++, Python, Matplotlib, Pandas} % Affiliation/role
    {Exploring Game Flow - Boss'n'Run} % Organization/group
    {\href{https://brandnerkasper.itch.io/bossn-run}{\faGamepad}} % HyperLink github
    {Sep. 2022 - Jun. 2023} % Date(s)
    {
      \begin{cvitems} % Description(s) of experience/contributions/knowledge
        \item {Build a 3D Jump'n'Run prototype with two colleagues in Unreal Engine 5, where the level layout is procedurally generated based on the movement parameters of the player character.}
        \item {Designed, animated and rigged the player character (Blender) and implemented multiple different movement mechanics, e.g. climbing.}
        \item {Collected movement data from different 3D jump'n'runs, e.g. Super Mario 64, and analyzed the data in 3D plots (Matplotlib \& Python).}%Based on this work I collected data from different famous 3D jump'n'runs, e.g. Super Mario 64, analyzed the data in 3D plots using Matplotlib & Python.}
        \item {Presented this work at the Conference of Games (CoG) in Milan - 2024.}
      \end{cvitems}
    }
    
%---------------------------------------------------------
%   \cventry
%     {Godot, SQLite} % Affiliation/role
%     {Crowd sourcing Help Facilites} % Organization/group
%     {\href{https://github.com/BrandnerKasper/CrowdbasedHelpFacilites}{\faGithubSquare}} % HyperLink github
%     {Mar. 2021 - Sep. 2021} % Date(s)
%     {
%       \begin{cvitems} % Description(s) of experience/contributions/knowledge
%         \item {Developed a plugin (GDScript) to display, create and vote on tips about the Godot Engine.}
%         \item {Created the UI (Figma) and the database (SQLite).}
%       \end{cvitems}
%     }
% %---------------------------------------------------------

  % \cventry
  %   {C++, CMake, OpenGL} % Affiliation/role
  %   {Eternal Game Engine} % Organization/group
  %   {\href{https://github.com/BrandnerKasper/Eternal}{\faGithubSquare}} % HyperLink github
  %   {Mar. 2020 - Mar. 2021} % Date(s)
  %   {
  %     \begin{cvitems} % Description(s) of experience/contributions/knowledge
  %       \item {Build a game engine in C++ with its own editor to make a simple 2D jump'n'run about leading a duck egg back into its nest.}
  %       \item {Focused on implementing an extensive editor (Dear ImGui) to design and save my levels as .yaml files.}
  %     \end{cvitems}
  %   }

%---------------------------------------------------------
\end{cventries}
