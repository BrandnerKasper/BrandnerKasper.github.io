%%%%%%%%%%%%%%%%%%%%%%%%%%%%%%%%%%%%%%%%%
% Important note:
% This template requires the resume.cls file to be in the same directory as the
% .tex file. The resume.cls file provides the resume style used for structuring the
% document.
%
%%%%%%%%%%%%%%%%%%%%%%%%%%%%%%%%%%%%%%%%%

%----------------------------------------------------------------------------------------
%	PACKAGES AND OTHER DOCUMENT CONFIGURATIONS
%----------------------------------------------------------------------------------------

\documentclass{resume} % Use the custom resume.cls style

\usepackage[left=0.75in,top=0.6in,right=0.75in,bottom=0.6in]{geometry} % Document margins
\usepackage[colorlinks=true, linkcolor=blue, urlcolor=blue]{hyperref}
\newcommand{\tab}[1]{\hspace{.2667\textwidth}\rlap{#1}}
\newcommand{\itab}[1]{\hspace{0em}\rlap{#1}}
\name{CV - Tobias Brandner} % Your name
\address{Semmelstraße 9, 97070 Würzburg} % Your address
%\address{123 Pleasant Lane \\ City, State 12345} % Your secondary addess (optional)
\address{(+49) 1798298854 \\ tobias.brandner@gmx.de} % Your phone number and email

\begin{document}

%----------------------------------------------------------------------------------------
%	EDUCATION SECTION
%----------------------------------------------------------------------------------------
\begin{rSection}{Education}
%--copy and paste this region  if you need more--
{\bf Msc in Computer Science - Specialization in Artificial Intelligence} \hfill {1.5} 
\\ Julius-Maximilian-University Würzburg \hfill {\em Arpil 2021 - September 2024}

{\bf Bsc in Games Engineering} \hfill {1.8} 
\\ Julius-Maximilian-University Würzburg \hfill {\em October 2017 - September 2021}
%--copy and paste this region  if you need more--
\end{rSection}

%----------------------------------------------------------------------------------------
%	EXPERIENCE SECTION
%----------------------------------------------------------------------------------------
\begin{rSection}{Experience}
%--copy and paste this region  if you need more--
{\bf Research Assistant}{, Julius-Maximilian-University Würzburg} \hfill {\em November 2021 - August 2023}\\
{\bf Technologies:} Unity, C\# 
\begin{itemize}
    \item Worked on the open-source framework ViaVR which creates VR apps for medical treatment.
\end{itemize}
%--copy and paste this region  if you need more--
{\bf Teaching Assistant}{, Julius-Maximilian-University Würzburg} \hfill {\em August 2021 - August 2023}\\
{\bf Technologies:} C++, OpenGL, CMake
\begin{itemize}
    \item Tutored a course on game engine development, teaching the basics of rendering, input handling and game loop.
\end{itemize}
%--copy and paste this region  if you need more--
{\bf Internship Software Developer}{, Gentle Troll Entertainment GmbH} \hfill {\em March 2021 - June 2021}\\
{\bf Technologies:} Unity, C\#
\begin{itemize}
    \item Developed a serious game for teaching children about management in sports.
\end{itemize}
\end{rSection}

%--------------------------------------------------------------------------------
%    PROJECTS
%-----------------------------------------------------------------------------------------------
\begin{rSection}{Projects}
%--copy and paste this region  if you need more--
{\bf Real-Time Rendering Super Resolution with Unreal Engine 5} \hfill {\href{https://github.com/BrandnerKasper/URTSR}{Github}} \\
{\bf Technologies:} Python, Pytorch, Unreal Engine 5
\begin{itemize}
    \item Developed a neural method to increase resolution from 1080p to 4k and image quality in real-time.
\end{itemize}

{\bf Abyssal Enigma - Dive In Edition} \hfill {\href{https://miggli.itch.io/abyssal-enigma/devlog/638746/abyssal-engima-dive-in-edition-vr}{Itchio}} \\
{\bf Technologies:} Unreal Engine 5, C++, VR, Blender
\begin{itemize}
    \item Developed a first person deep sea exploration game and ported it to VR.
\end{itemize}

{\bf Eternal Game Engine} \hfill {\href{https://github.com/BrandnerKasper/Eternal}{Github}} \\
{\bf Technologies:} C++, OpenGL, PreMake
\begin{itemize}
    \item Developed a game engine with OpenGL as render backend and an editor written with Dear ImGui.
\end{itemize}
%--copy and paste this region  if you need more--
\end{rSection}

%--------------------------------------------------------------------------------
%    Publications
%-----------------------------------------------------------------------------------------------
\begin{rSection}{Publications}
%--copy and paste this region  if you need more--
{\bf Analysis and Generation of Flow in 3D Jump’n’Run Games.} \hfill {\href{https://ieeexplore.ieee.org/abstract/document/10645536}{PDF}}\\
2024 IEEE Conference on Games (CoG).\\
Tobias Brandner, Marc Mußmann, and Sebastian von Mammen.\\

%{\bf Investigating Crowdsourced Help Facilities for Enhancing User Guidance.} \hfill {\href{https://diglib.eg.org/server/api/core/bitstreams/f635ab29-9038-4e8f-8563-89c7def8902f/content}{PDF}}\\
%2023 IMET. \\
%Sooraj Babu, Tobias Brandner, Samuel Truman, Sebastian von Mammen. \\\\
\end{rSection}

%----------------------------------------------------------------------------------------
%	SKILLS SECTION
%----------------------------------------------------------------------------------------
\begin{rSection}{Skills}
    \begin{tabular}{>{\bfseries}l l} % Bold the first column
        Languages: & Python, C++, C\#, Java, Rust \\
        Frameworks/Libraries: & Pytorch, Matplotlib, Pandas, OpenGL \\
        Game Engines: & Unreal, Unity, Godot \\
        Tools: & Git, CMake, Blender \\
    \end{tabular}
\end{rSection}

\end{document}----------------------------

